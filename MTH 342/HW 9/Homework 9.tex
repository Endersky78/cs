\documentclass[12pt, letterpaper]{article}
\setlength{\parindent}{0cm}
\usepackage[utf8]{inputenc}
\usepackage{amsmath}
\usepackage{amsthm}
\usepackage{amssymb}
\usepackage{mathtools}

\title{Homework 9}
\author{Jackson Hart}
\date{March 11th, 2022}

\begin{document}

\maketitle

\section*{Problem 1}

\[ A = \begin{bmatrix} 0 & 2 & 2 \\ 2 & 0 & 2 \\ 2 & 2 & 0 \end{bmatrix} \]
\[ det(A - \lambda I) = \begin{vmatrix} -\lambda & 2 & 2 \\ 2 & -\lambda & 2 \\ 2 & 2 & -\lambda \end{vmatrix} = -\lambda^3 + 12 \lambda + 16 = -(\lambda + 2)^2(\lambda - 4) \]

So we have that $A$ has eigenvalues $\lambda_1 = -2$ and $\lambda_2 = 4$. With $\lambda_1$ having a algebraic multiplicity of 2. Now, to calculate the eigenvectors. 

\[ A - \lambda_1 = \begin{bmatrix} 2 & 2 & 2 \\ 2 & 2 & 2 \\ 2 & 2 & 2 \end{bmatrix} \xrightarrow[]{REF} \begin{bmatrix} 1 & 1 & 1 \\ 0 & 0 & 0 \\ 0 & 0 & 0 \end{bmatrix} \]

Which gives us the eigenvectors $v_1 = \begin{bmatrix} -1 \\ 1 \\ 0 \end{bmatrix}$ and $v_2 = \begin{bmatrix} -1 \\ 0 \\ 1 \end{bmatrix}$. 

\[ A - \lambda_2 = \begin{bmatrix} -4 & 2 & 2 \\ 2 & -4 & 2 \\ 2 & 2 & -4 \end{bmatrix} \xrightarrow[]{REF} \begin{bmatrix} 1 & 0 & -1 \\ 0 & 1 & -1 \\ 0 & 0 & 0 \end{bmatrix} \] 

Which gives us the eigenvector $v_3 = \begin{bmatrix} 1 \\ 1 \\ 1 \end{bmatrix}$. 

Now we must normalize the eigenvectors. Since all the values we have seen thus far are real, I will assume we are in the space of real numbers and will use the dot product.

\[ ||v_1||^2 = \begin{bmatrix} -1 \\ 1 \\ 0 \end{bmatrix} \cdot \begin{bmatrix} -1 \\ 1 \\ 0 \end{bmatrix} = 2 \]

So our first orthonormal eigenvector $u_1 = \frac{1}{\sqrt{2}} \begin{bmatrix} -1 \\ 1 \\ 0 \end{bmatrix}$

\[ ||v_2||^2 = \begin{bmatrix} -1 \\ 0 \\ 1 \end{bmatrix} \cdot \begin{bmatrix} -1 \\ 0 \\ 1 \end{bmatrix} = 2 \]

\[ ||v_3||^2 = \begin{bmatrix} 1 \\ 1 \\ 1 \end{bmatrix} \cdot \begin{bmatrix} 1 \\ 1 \\ 1 \end{bmatrix} = 3 \]

So then we have that $u_2 = \frac{1}{\sqrt{2}} \begin{bmatrix} -1 \\ 0 \\ 1 \end{bmatrix} \text{ and } u_3 = \frac{1}{\sqrt{3}} \begin{bmatrix} 1 \\ 1 \\ 1 \end{bmatrix}$. 

So then we have that

\[ U = \begin{bmatrix} -\frac{1}{\sqrt{2}} & -\frac{1}{\sqrt{2}} & \frac{1}{\sqrt{3}} \\ \frac{1}{\sqrt{2}} & 0 & \frac{1}{\sqrt{3}} \\ 0 & \frac{1}{\sqrt{2}} & \frac{1}{\sqrt{3}} \end{bmatrix}  \]

And because this is real numbers

\[ U^* = U^T = \begin{bmatrix} -\frac{1}{\sqrt{2}} & \frac{1}{\sqrt{2}} & 0 \\ -\frac{1}{\sqrt{2}} & 0 & \frac{1}{\sqrt{2}} \\ \frac{1}{\sqrt{3}} & \frac{1}{\sqrt{3}} & \frac{1}{\sqrt{3}} \end{bmatrix} \]

So,

\[ A = \begin{bmatrix} -\frac{1}{\sqrt{2}} & -\frac{1}{\sqrt{2}} & \frac{1}{\sqrt{3}} \\ \frac{1}{\sqrt{2}} & 0 & \frac{1}{\sqrt{3}} \\ 0 & \frac{1}{\sqrt{2}} & \frac{1}{\sqrt{3}} \end{bmatrix} \begin{bmatrix} -2 & 0 & 0 \\ 0 & -2 & 0 \\ 0 & 0 & 4 \end{bmatrix} \begin{bmatrix} -\frac{1}{\sqrt{2}} & \frac{1}{\sqrt{2}} & 0 \\ -\frac{1}{\sqrt{2}} & 0 & \frac{1}{\sqrt{2}} \\ \frac{1}{\sqrt{3}} & \frac{1}{\sqrt{3}} & \frac{1}{\sqrt{3}} \end{bmatrix} \]

\section*{Problem 2}
Let $A$ and $B$ be unitarily equivalent $n \times n$ matrices.

\subsection*{A)}
$\text{trace}(A^*A) = \text{trace}(B^*B)$. 

\begin{proof}
If $A$ and $B$ are unitarily equivalent, then $\exists U$ such that $U$ is unitary, $A = UBU^*$, and similarly $B = U^*AU$. So then we have that

\[ \text{trace}(A^*A) = \text{trace}(A^*UBU^*) = \text{trace}(U^*A^*UB) = \text{trace}(B^*B) \]
\end{proof}

\subsection*{B)}

\[ \sum^n_{j,k=1} |A_{j,k}|^2 = \sum^n_{j,k=1} |B_{j,k}|^2 \]

\begin{proof}
Consider the matrix representation of $A^*A$ and its two first elements on its main diagonal, $A_{1,1}, A_{2,2}$. By matrix multiplication, we can write them as

\[ A_{1,1} = \sum^n_{k=1} A_{k,1} \overline{A_{k,1}} \]
\[ A_{2,2} = \sum^n_{k=1} A_{k,2} \overline{A_{k,2}} \]

The coordinates are not flipped due to $A^*$ being the complex conjugate of the \textbf{transpose}. So then, the sum of $A^*A$'s elements on the main diagonal can be written as

\[ \text{trace}(A^*A) = \sum^n_{j,k=1} A_{k,j} \overline{A_{k,j}} = \sum^n_{j,k=1} |A_{k,j}|^2 \]

The same can be said for $B$. $A$ and $B$ have already been defined to be unitarily equivalent, and thus

\[ \text{trace}(A^*A) = \sum^n_{j,k=1} |A_{k,j}|^2 \]
\[ \text{trace}(B^*B) = \sum^n_{j,k=1} |B_{k,j}|^2 \]
\[ \sum^n_{j,k=1} |A_{k,j}|^2 = \text{trace}(A^*A) = \text{trace}(B^*B) = \sum^n_{j,k=1} |B_{k,j}|^2 \]
\[ \sum^n_{j,k=1} |A_{k,j}|^2 = \sum^n_{j,k=1} |B_{k,j}|^2 \]

\end{proof}

\subsection*{C)}

\[ \text{let } A = \begin{bmatrix} 1 & 2 \\ 2 & i \end{bmatrix} \]
\[ \text{let } B = \begin{bmatrix} i & 4 \\ 1 & 1 \end{bmatrix} \]

\[ \sum^n_{j,k=1} |A_{k,j}|^2 = 1^2 + 2^2 + 2^2 + |i|^2 = 10 \]
\[ \sum^n_{j,k=1} |B_{k,j}|^2 = |i|^2 + 4^2 + 1^2 + 1^2 = 19 \]
\[ \sum^n_{j,k=1} |A_{k,j}|^2 \neq \sum^n_{j,k=1} |B_{k,j}|^2 \]

$\therefore A \text{ and } B$ are not unitarily equivalent. 

\section*{Problem 3}

\subsection*{A)}
trace$\begin{bmatrix} 1 & 0 \\ 0 & 1 \end{bmatrix} = 2$ and trace$\begin{bmatrix} 0 & 1 \\ 1 & 0 \end{bmatrix} = 0$. So these are not unitarily equivalent. 

\subsection*{B)}

\[ \text{let } A = \begin{bmatrix} 0 & 1 \\ 1 & 0 \end{bmatrix} \]
\[ \text{let } B = \begin{bmatrix} 0 & 1/2 \\ 1/2 & 0 \end{bmatrix} \]

\[ \sum^n_{j,k=1} |A_{k,j}|^2 = 0^2 + 1^2 + 1^2 + 0^2 = 2 \]
\[ \sum^n_{j,k=1} |B_{k,j}|^2 = 0^2 + \frac{1}{2}^2 + \frac{1}{2}^2 + 0^2 = \frac{1}{2} \]
\[ \sum^n_{j,k=1} |A_{k,j}|^2 \neq \sum^n_{j,k=1} |B_{k,j}|^2 \]

$\therefore A \text{ and } B$ are not unitarily equivalent.

\subsection*{C)}
\[ \begin{vmatrix} 0 & 1 & 0 \\ -1 & 0 & 0 \\ 0 & 0 & 1 \end{vmatrix} = 1 \]
\[ \begin{vmatrix} 2 & 0 & 0 \\ 0 & -1 & 0 \\ 0 & 0 & 0 \end{vmatrix} = 0 \]

Unitarily equivalent matrices share determinants, so these cannot be unitarily equivalent. 

\subsection*{D)}
\[ \text{let } A = \begin{bmatrix} 0 & 1 & 0 \\ -1 & 0 & 0 \\ 0 & 0 & 1 \end{bmatrix} \]
\[ \text{let } B = \begin{bmatrix} 1 & 0 & 0 \\ 0 & -i & 0 \\ 0 & 0 & i \end{bmatrix} \]

\[ \text{det}(A - \lambda I) = \begin{vmatrix} -\lambda & 1 & 0 \\ -1 & -\lambda & 0 \\ 0 & 0 & 1 - \lambda \end{vmatrix} = -(\lambda - 1)(\lambda^2 + 1) \]

So we have the eigenvalues of $A$, $\lambda_1 = 1, \lambda_2 = -i, \text{ and } \lambda_3 = i$. Solving $A - \lambda_i I = 0$ for all $\lambda$s gives us the eigenvectors, 

\[ v_1 = \begin{bmatrix} 0 \\ 0 \\ 1 \end{bmatrix} \]
\[ v_2 = \begin{bmatrix} i \\ 1 \\ 0 \end{bmatrix} \]
\[ v_3 = \begin{bmatrix} -i \\ 1 \\ 0 \end{bmatrix} \]

\[ \langle v_1, v_2 \rangle = 0 \]
\[ \langle v_1, v_3 \rangle = 0 \]
\[ \langle v_2, v_3 \rangle = 0 \]

So the set of these eigenvectors is orthogonal, and thus are also linearly independent.

\[ ||v_1|| = \sqrt{1^2} = 1 \]
\[ ||v_2|| = \sqrt{(i \times -i)^2 + 1^2 + 0} = \sqrt{2} \]
\[ ||v_3|| = \sqrt{(-i \times i)^2 + 1^2 + 0} = \sqrt{2} \]

We must normalize these vectors to get our orthonormal basis, so we have that

\[ \mathcal{B} = \begin{Bmatrix} \begin{bmatrix} 0 \\ 0 \\ 1 \end{bmatrix}, \frac{1}{\sqrt{2}} \begin{bmatrix} i \\ 1 \\ 0 \end{bmatrix}, \frac{1}{\sqrt{2}} \begin{bmatrix} -i \\ 1 \\ 0 \end{bmatrix} \end{Bmatrix} \]

with the corresponding vectors $u_1, u_2, u_3$. This proves that $A$ is unitarily equivalent to a diagonal matrix, and so we have that

\[ A = \frac{1}{\sqrt{2}} \begin{bmatrix} 0 & i & -i \\ 0 & 1 & 1 \\ \sqrt{2} & 0 & 0 \end{bmatrix} \begin{bmatrix} 1 & 0 & 0 \\ 0 & -i & 0 \\ 0 & 0 & i \end{bmatrix} \frac{1}{\sqrt{2}} \begin{bmatrix} 0 & 0 & \sqrt{2} \\ -i & 1 & 0 \\ i & 1 & 0 \end{bmatrix} \]

And therefore, $A$ is unitarily equivalent to $B$. 

\subsection*{E)}
\[ \text{let } A = \begin{bmatrix} 1 & 1 & 0 \\ 0 & 2 & 2 \\ 0 & 0 & 3 \end{bmatrix} \]
\[ \text{let } B = \begin{bmatrix} 1 & 0 & 0 \\ 0 & 2 & 0 \\ 0 & 0 & 3 \end{bmatrix} \]

\[ \sum^n_{j,k=1} |A_{k,j}|^2 = 1^2 + 1^2 + 2^2 + 2^2 + 3^2 = 19 \]
\[ \sum^n_{j,k=1} |B_{k,j}|^2 = 1^2 + 2^2 + 3^2 = 14 \]

\[ \sum^n_{j,k=1} |A_{k,j}|^2 \neq \sum^n_{j,k=1} |B_{k,j}|^2 \]

$\therefore A \text{ and } B$ are not unitarily equivalent.

\section*{Problem 4}
Suppose $A$ is a normal operator on an inner product space $V$ and that 3 and 4 are eigenvalues of $A$. Then there exists a vector $v \in V$ such that $||v|| = \sqrt{2}$ and $||Av|| = 5$. 

\begin{proof}
Let $v_1, v_2 \in V$ be eigenvectors corresponding to 3 and 4. Because $A$ is a normal operator, $A$ has an orthonormal basis of eigenvectors which means

\[ ||v_1|| = 1 \]
\[ ||v_2|| = 1 \]

By the Generalized Pythagorean identity,

\[ ||v_1 + v_2|| = \sqrt{1^2 + 1^2} = \sqrt{2} \]

From this, we get

\[ ||Av_1 + Av_2|| = ||3v_1 + 4v_2|| = \sqrt{3^2 + 4^2} = 5 \]

\end{proof}

\section*{Problem 5}
Suppose $V$ is a complex inner product space and $T$ is a normal operator on $V$ such that $T^7 = T^6$. Then $T$ is self-adjoint and $T^2 = T$.

\begin{proof}
By the Spectral Theorem, $V$ has a orthonormal basis of eigenvectors of $T$, $\mathcal{B} = e_1, ... , e_n$. Let $\lambda_1, ... , \lambda_n$ be the corresponding eigenvalues. Then

\[ Te_i = \lambda_ie_i \forall i = 1, ... , n \]

Then we have that

\[ T^7e_i = \lambda^7_ie_i \text{ and } T^6e_i = \lambda^6_ie_i \] 

Which gives $\lambda_i$ the possibility of either being 0 or 1. Because of this, $[T]_{\mathcal{B} \mathcal{B}}$ is a diagonal matrix with eigenvalues on the diagonal. Because the $[T]^*_{\mathcal{B} \mathcal{B}}$ is the conjugate of the transpose, and $[T]_{\mathcal{B} \mathcal{B}}$ is a diagonal matrix with only real values, $[T]^*_{\mathcal{B} \mathcal{B}} = [T]_{\mathcal{B} \mathcal{B}}$. Knowing this,

\[ T^2e_i = \lambda^2_ie_i \rightarrow \text{ because $\lambda_i$ is either 0 or 1, $\lambda^2_i$ is either 0 or 1, so...} \]

\[ T^2e_i = \lambda_ie_i = Te_i \]
\end{proof}

\end{document}